\documentclass[DM,lsstdraft,STR,toc]{lsstdoc}
\usepackage{geometry}
\usepackage{longtable,booktabs}
\usepackage{enumitem}
\usepackage{arydshln}

\input meta.tex

\providecommand{\tightlist}{
  \setlength{\itemsep}{0pt}\setlength{\parskip}{0pt}}

\begin{document}

\def\milestoneName{Alert Distribution Validation}
\def\milestoneId{LDM-503-5:}
\def\product{Prompt}

\setDocCompact{true}

\title[\milestoneId{}~Test Report]{\milestoneId{} (\milestoneName{})~Test Plan and Report}
\setDocRef{\lsstDocType-\lsstDocNum}
\setDocDate{\vcsdate}
\setDocUpstreamLocation{\url{https://github.com/lsst/lsst-texmf/examples}}
\author{ Eric Bellm }

\input history_and_info.tex


\setDocAbstract{
This is the test plan and report for \milestoneId{} (\milestoneName{}), an LSST level 2 milestone pertaining to the \product{} component provided by the Data Management Subsystem.
}


\maketitle

\section{Introduction}
\label{sect:intro}


\subsection{Objectives}
\label{sect:objectives}

This test activity demonstrates the successful execution of the major
components of the Alert Distribution system at the scale of LSST
Operations.\\[2\baselineskip]It will demonstrate that:\\

\begin{itemize}
\tightlist
\item
  The Alert Distribution payloads can be made available on systems
  managed by the LSST Data Facility
\item
  The Alert Distribution payloads can be executed under the control of
  the relevant LDF services
\item
  All required science data products can be passed through the
  components of the alert distribution service.
\end{itemize}

Note that this test plan does not extend to testing integration of the
Alert Distribution System with the Alert Generation Science Pipeline or
the LSST Science Platform.\\
We also do not test the Alert Database at this time as its functional
requirements are not sufficiently specified to enable
testing.\\[2\baselineskip]

\subsection{Scope}\label{scope}

The overall test plan for the LSST Data Management system is described
in \citeds{LDM-503}.\\
Success in this test plan is intended to demonstrate completion of the
milestone LDM-503-05.\\
The overall LSST Alert Distribution System test specification is defined
in \citeds{LDM-533}.



\subsection{System Overview}
\label{sect:systemoverview}

The LSST Alert Distribution Service is that part of the LSST Data
Management System which will be responsible for realtime alert
distribution and filtering during LSST operations. The LDM-503-5
milestone exercises only the distribution and filtering aspects; it does
not consider generation of alerts by the LSST Science Pipelines or the
user interfaces to the simple filtering service through the LSST Science
Platform.\\[2\baselineskip]

\subsection{Applicable Documents}\label{applicable-documents}

\citeds{LDM-294} Data Management Organization and Management\\
\citeds{LDM-503} Data Management Test Plan\\
\citeds{LDM-533} LSST Level 1 System Test Specification


\subsection{Document Overview}
\label{sect:docoverview}

This document was generated from Jira, obtaining the relevant information from the 
\href{https://jira.lsstcorp.org/secure/Tests.jspa#/testPlan/LVV-P1}{LVV-P1}
~Jira Test Plan and related Test Cycles (
  \href{https://jira.lsstcorp.org/secure/Tests.jspa#/testCycle/LVV-C3}{LVV-C3}
).

%The following general sections are completed before the start of the test activity.

Section \ref{sect:intro} provides an overview of the test campaign, the system under test (\product{}), the applicable documentation, and explains how this document is organized.
Section \ref{sect:configuration}  describes the configuration used for this test.
Section \ref{sect:personnel} lists all the individuals involved and describes their roles.
%Section \ref{sect:plannedtestactivities} provides the list of planned test cycles and test cases, including all relevant information that fully describes the test campaign.

Section \ref{sect:overview} provides a summary of the test results, including an overview in Table \ref{table:summary}, an overall assessment statement and suggestions for possible improvements.
Section \ref{sect:detailedtestresults} provides detailed results for each step in each test case.

The current status of test plan LVV-P1 in Jira is Draft.

\subsection{References}
\label{sect:references}
\renewcommand{\refname}{}
\bibliography{lsst,refs,books,refs_ads}
\section{Test Configuration}
\label{sect:configuration}

\subsection{Data Collection}

  Observing is not required for this test campaign.

\subsection{Verification Environment}
\label{sect:hwconf}
  This test activity shall be executed on the Kubernetes Commons at the
LDF.\\
As discussed in https://dmtn-028.lsst.io/ and https://dmtn-081.lsst.io/,
the test machine should have at least 16 cores, 64 GB of memory and
access to at least 1.5 TB of shared storage.





\section{Personnel}
\label{sect:personnel}

The following personnel are involved in this test activity:

\begin{itemize}
\item Test Plan (LVV-P1) owner: Eric Bellm
\item Test Cycles:
\begin{itemize}
  \item LVV-C3 owner: 
    Eric Bellm
  \begin{itemize}
    \item Test case LVV-T216 tester: Eric Bellm
    \item Test case LVV-T217 tester: Eric Bellm
    \item Test case LVV-T218 tester: Eric Bellm
  \end{itemize}
\end{itemize}
\item Additional Test Personnel involved: None
\end{itemize}

\newpage

\section{Overview of the Test Results}
\label{sect:overview}

\subsection{Summary}
\label{sect:summarytable}

\begin{longtable} {p{0.2\textwidth}p{0.2\textwidth}p{0.6\textwidth}}
\toprule
  \multicolumn{3}{c}{ Test Cycle {\bf LVV-C3: LDM-503-5: Alert Distribution Validation }} \\\hline
  {\bf \footnotesize test case} & {\bf \footnotesize status} & {\bf \footnotesize comment} \\\toprule
    \href{https://jira.lsstcorp.org/secure/Tests.jspa#/testCase/LVV-T216}{LVV-T216} 
    & Conditional Pass & Missing git-lfs in steps 2 and 3. Used pre-built Docker image instead.
Remaining steps pass.

    \\\hline
    \href{https://jira.lsstcorp.org/secure/Tests.jspa#/testCase/LVV-T217}{LVV-T217} 
    & Conditional Pass & Missing git-lfs in step 2. Used docker image instead. Remaining steps
pass.

    \\\hline
    \href{https://jira.lsstcorp.org/secure/Tests.jspa#/testCase/LVV-T218}{LVV-T218} 
    & Conditional Pass & Missing git-lfs in step 2 and 3. Used docker image instead. Last step
result deviated sligthly from the expected result for one consumer.

    \\\hline

\caption{Test Results Summary}
\label{table:summary}
\end{longtable}

\subsection{Overall Assessment}
\label{sect:overallassessment}

The overall result of the test campaign is:
PASS.\\[2\baselineskip]Missing elements on the Kubernetes Commons
prevented us from building the Docker images in place and using a
complete night of unique alerts. However, using pre-built Docker images
with a smaller, repeated sample of alerts enabled the test to proceed as
expected.


\subsection{Recommended Improvements}
\label{sect:recommendations}

Re-execution with the missing components on the Kubernetes
Commons.\\[2\baselineskip]Examination of the deserialized alerts in the
logs is cumbersome and does not allow real verification of the alert
content. Improved tools for counting alerts, comparing their contents to
the sent alerts, and timing throughput would all improve the utility of
this test.


\newpage
\section{Detailed Test Results}
\label{sect:detailedtestresults}


  \subsection{Test Cycle LVV-C3 }

Open test cycle {\it \href{https://jira.lsstcorp.org/secure/Tests.jspa#/testrun/LVV-C3}{LDM-503-5: Alert Distribution Validation}} in Jira.

  LDM-503-5: Alert Distribution Validation\\
  Status: Done

  Test run with test cases in draft.


  \subsubsection{Software Version/Baseline}
    This test uses only the code in\\
https://github.com/lsst-dm/alert\_stream/ , commit
ef3c9e16c42a3265d4f6258ba6983be8b009e7c0


  \subsubsection{Configuration}
    \textbf{Documents}\\[2\baselineskip]This test report refers to the
execution of tests described in the section 1.2 and documented in the
\citeds{LDM-533} test specification issue
2.0.\\[2\baselineskip]\textbf{Hardware}\\[2\baselineskip]This test
activity shall be executed on the Kubernetes Commons at the LDF. As
discussed in https://dmtn-028.lsst.io/ and https://dmtn-081.lsst.io/,
the test machine should have at least 16 cores, 64 GB of memory and
access to at least 1.5 TB of shared
storage.\\[2\baselineskip]\textbf{Software}\\[2\baselineskip]This test
uses only the code in\\
https://github.com/lsst-dm/alert\_stream/ , commit
ef3c9e16c42a3265d4f6258ba6983be8b009e7c0\\[2\baselineskip]\textbf{Input
Data}\\[2\baselineskip]Input data for all tests was based on simulated
alert packets\\[2\baselineskip]


  \subsubsection{Test Cases in LVV-C3 Test Cycle}


    \paragraph{Test Case LVV-T216 }\mbox{}\\

Open  \href{https://jira.lsstcorp.org/secure/Tests.jspa#/testCase/LVV-T216}{\textit{ LVV-T216 } }
test case in Jira.

    This test will check:\\

\begin{itemize}
\tightlist
\item
  That the Alert Distribution payloads are available from documented
  channels.
\item
  That the Alert Distribution payloads can be installed on LSST Data
  Facility-managed systems.
\item
  That the Alert Distribution payloads can be executed by LSST Data
  Facility-managed systems.
\end{itemize}


    {\bf Preconditions}:\\
    

    Execution status: {\bf Conditional Pass }

    Final comment:\\Missing git-lfs in steps 2 and 3. Used pre-built Docker image instead.
Remaining steps pass.



    Detailed step results:

    \begin{longtable}{p{1cm}p{2cm}p{13cm}}
    \hline
    {Step} & \multicolumn{2}{c}{Description, Results and Status}\\ \hline
      1 & Description &

      \begin{minipage}[t]{13cm}{\footnotesize
      Download Kafka Docker image from
https://github.com/lsst-dm/alert\_stream.

      \vspace{\dp0}
      } \end{minipage} \\
      \\ \cdashline{2-3}

      & Expected Result & 

      \begin{minipage}[t]{13cm}{\footnotesize
      Runs without error

      \vspace{\dp0}
      } \end{minipage} \\
      \\ \cdashline{2-3}

      & \begin{minipage}[t]{2cm}{Actual\\ Result}\end{minipage}   & 
      \begin{minipage}[t]{13cm}{\footnotesize
      runs without error

      \vspace{\dp0}
      } \end{minipage} \\
      \\ \cdashline{2-3}

      & Status          & Pass \\ \hline

      2 & Description &

      \begin{minipage}[t]{13cm}{\footnotesize
      Change to the alert\_stream directory and build the docker image.\\

\begin{verbatim}
docker build -t "lsst-kub001:5000/alert_stream" .
\end{verbatim}

      \vspace{\dp0}
      } \end{minipage} \\
      \\ \cdashline{2-3}

      & Expected Result & 

      \begin{minipage}[t]{13cm}{\footnotesize
      Runs without error

      \vspace{\dp0}
      } \end{minipage} \\
      \\ \cdashline{2-3}

      & \begin{minipage}[t]{2cm}{Actual\\ Result}\end{minipage}   & 
      \begin{minipage}[t]{13cm}{\footnotesize
      Did not execute because of missing git-lfs; used pre-built Docker image
instead.

      \vspace{\dp0}
      } \end{minipage} \\
      \\ \cdashline{2-3}

      & Status          & Conditional Pass \\ \hline

      3 & Description &

      \begin{minipage}[t]{13cm}{\footnotesize
      Register it with Kubernetes\\[2\baselineskip]docker push
lsst-kub001:5000/alert\_stream

      \vspace{\dp0}
      } \end{minipage} \\
      \\ \cdashline{2-3}

      & Expected Result & 

      \begin{minipage}[t]{13cm}{\footnotesize
      Runs without error

      \vspace{\dp0}
      } \end{minipage} \\
      \\ \cdashline{2-3}

      & \begin{minipage}[t]{2cm}{Actual\\ Result}\end{minipage}   & 
      \begin{minipage}[t]{13cm}{\footnotesize
      Did not execute because of missing git-lfs; used pre-built Docker image
instead

      \vspace{\dp0}
      } \end{minipage} \\
      \\ \cdashline{2-3}

      & Status          & Conditional Pass \\ \hline

      4 & Description &

      \begin{minipage}[t]{13cm}{\footnotesize
      From the alert\_stream/kubernetes directory, start Kafka and
Zookeeper:\\[2\baselineskip]

\begin{verbatim}
kubectl create -f zookeeper-service.yaml
kubectl create -f zookeeper-deployment.yaml
kubectl create -f kafka-deployment.yaml
kubectl create -f kafka-service.yaml
\end{verbatim}

(use kubectl get pods/services between each command to check status;
wait until each is ``Running'' before starting the next
command)\\[2\baselineskip]

      \vspace{\dp0}
      } \end{minipage} \\
      \\ \cdashline{2-3}

      & Expected Result & 

      \begin{minipage}[t]{13cm}{\footnotesize
      Runs without error

      \vspace{\dp0}
      } \end{minipage} \\
      \\ \cdashline{2-3}

      & \begin{minipage}[t]{2cm}{Actual\\ Result}\end{minipage}   & 
      \begin{minipage}[t]{13cm}{\footnotesize
      Runs without error\\[2\baselineskip]

      \vspace{\dp0}
      } \end{minipage} \\
      \\ \cdashline{2-3}

      & Status          & Pass \\ \hline

      5 & Description &

      \begin{minipage}[t]{13cm}{\footnotesize
      Confirm Kafka and Zookeeper are listed when
running\\[2\baselineskip]kubectl get
pods\\[2\baselineskip]and\\[2\baselineskip]kubectl get services

      \vspace{\dp0}
      } \end{minipage} \\
      \\ \cdashline{2-3}

      & Expected Result & 

      \begin{minipage}[t]{13cm}{\footnotesize
      Output should be similar to:\\[2\baselineskip]kubectl get pods\\
NAME ~ ~ ~ ~ ~ ~ ~ ~ ~ ~ ~ ~READY ~ ~ STATUS ~ ~RESTARTS ~ AGE\\
kafka-768ddf5564-xwgvh ~ ~ ~1/1 ~ ~ ~ Running ~ 0 ~ ~ ~ ~ ~31s\\
zookeeper-f798cc548-mgkpn ~ 1/1 ~ ~ ~ Running ~ 0 ~ ~ ~ ~
~1m\\[2\baselineskip]kubectl get services\\
NAME ~ ~ ~ ~TYPE ~ ~ ~ ~CLUSTER-IP ~ ~ ~EXTERNAL-IP ~ PORT(S) ~ ~ AGE\\
kafka ~ ~ ~ ClusterIP ~ 10.105.19.124 ~ \textless{}none\textgreater{} ~
~ ~ ~9092/TCP ~ ~6s\\
zookeeper ~ ClusterIP ~ 10.97.110.124 ~ \textless{}none\textgreater{} ~
~ ~ ~32181/TCP ~ 2m

      \vspace{\dp0}
      } \end{minipage} \\
      \\ \cdashline{2-3}

      & \begin{minipage}[t]{2cm}{Actual\\ Result}\end{minipage}   & 
      \begin{minipage}[t]{13cm}{\footnotesize
      as expected

      \vspace{\dp0}
      } \end{minipage} \\
      \\ \cdashline{2-3}

      & Status          & Pass \\ \hline

    \end{longtable}


    \paragraph{Test Case LVV-T217 }\mbox{}\\

Open  \href{https://jira.lsstcorp.org/secure/Tests.jspa#/testCase/LVV-T217}{\textit{ LVV-T217 } }
test case in Jira.

    This test will check that the full stream of LSST alerts can be
distributed to end users.\\[2\baselineskip]Specifically, this will
demonstrate that:

\begin{itemize}
\tightlist
\item
  Serialized alert packets can be loaded into the alert distribution
  system at LSST-relevant scales (10,000 alerts every 39 seconds);
\item
  Alert packets can be retrieved from the queue system at LSST-relevant
  scales.
\end{itemize}


    {\bf Preconditions}:\\
    Input data: A sample of Avro-formatted alert packets.


    Execution status: {\bf Conditional Pass }

    Final comment:\\Missing git-lfs in step 2. Used docker image instead. Remaining steps
pass.



    Detailed step results:

    \begin{longtable}{p{1cm}p{2cm}p{13cm}}
    \hline
    {Step} & \multicolumn{2}{c}{Description, Results and Status}\\ \hline
      1 & Description &

      \begin{minipage}[t]{13cm}{\footnotesize
      Download Kafka Docker image from
https://github.com/lsst-dm/alert\textbackslash{}\_stream.

      \vspace{\dp0}
      } \end{minipage} \\
      \\ \cdashline{2-3}

      & Expected Result & 

      \begin{minipage}[t]{13cm}{\footnotesize
      Runs without error

      \vspace{\dp0}
      } \end{minipage} \\
      \\ \cdashline{2-3}

      & \begin{minipage}[t]{2cm}{Actual\\ Result}\end{minipage}   & 
      \begin{minipage}[t]{13cm}{\footnotesize
      
      \vspace{\dp0}
      } \end{minipage} \\
      \\ \cdashline{2-3}

      & Status          & Pass \\ \hline

      2 & Description &

      \begin{minipage}[t]{13cm}{\footnotesize
      Change to the alert\_stream directory and build the docker image.\\

\begin{verbatim}
docker build -t "lsst-kub001:5000/alert_stream" .
\end{verbatim}

      \vspace{\dp0}
      } \end{minipage} \\
      \\ \cdashline{2-3}

      & Expected Result & 

      \begin{minipage}[t]{13cm}{\footnotesize
      Runs without error

      \vspace{\dp0}
      } \end{minipage} \\
      \\ \cdashline{2-3}

      & \begin{minipage}[t]{2cm}{Actual\\ Result}\end{minipage}   & 
      \begin{minipage}[t]{13cm}{\footnotesize
      Did not execute because of missing git-lfs; used pre-built Docker image
instead

      \vspace{\dp0}
      } \end{minipage} \\
      \\ \cdashline{2-3}

      & Status          & Conditional Pass \\ \hline

      3 & Description &

      \begin{minipage}[t]{13cm}{\footnotesize
      Register it with Kubernetes\\[2\baselineskip]docker push
lsst-kub001:5000/alert\_stream

      \vspace{\dp0}
      } \end{minipage} \\
      \\ \cdashline{2-3}

      & Expected Result & 

      \begin{minipage}[t]{13cm}{\footnotesize
      Runs without error

      \vspace{\dp0}
      } \end{minipage} \\
      \\ \cdashline{2-3}

      & \begin{minipage}[t]{2cm}{Actual\\ Result}\end{minipage}   & 
      \begin{minipage}[t]{13cm}{\footnotesize
      Did not execute because of missing git-lfs; used pre-built Docker image
instead

      \vspace{\dp0}
      } \end{minipage} \\
      \\ \cdashline{2-3}

      & Status          & Conditional Pass \\ \hline

      4 & Description &

      \begin{minipage}[t]{13cm}{\footnotesize
      From the alert\_stream/kubernetes directory, start Kafka and
Zookeeper:\\[2\baselineskip]

\begin{verbatim}
kubectl create -f zookeeper-service.yaml
kubectl create -f zookeeper-deployment.yaml
kubectl create -f kafka-deployment.yaml
kubectl create -f kafka-service.yaml
\end{verbatim}

(use kubectl get pods/services between each command to check status;
wait until each is ``Running'' before starting the next
command)\\[2\baselineskip]

      \vspace{\dp0}
      } \end{minipage} \\
      \\ \cdashline{2-3}

      & Expected Result & 

      \begin{minipage}[t]{13cm}{\footnotesize
      Runs without error

      \vspace{\dp0}
      } \end{minipage} \\
      \\ \cdashline{2-3}

      & \begin{minipage}[t]{2cm}{Actual\\ Result}\end{minipage}   & 
      \begin{minipage}[t]{13cm}{\footnotesize
      
      \vspace{\dp0}
      } \end{minipage} \\
      \\ \cdashline{2-3}

      & Status          & Pass \\ \hline

      5 & Description &

      \begin{minipage}[t]{13cm}{\footnotesize
      Confirm Kafka and Zookeeper are listed when
running\\[2\baselineskip]kubectl get
pods\\[2\baselineskip]and\\[2\baselineskip]kubectl get services

      \vspace{\dp0}
      } \end{minipage} \\
      \\ \cdashline{2-3}

      & Expected Result & 

      \begin{minipage}[t]{13cm}{\footnotesize
      Output should be similar to:\\[2\baselineskip]kubectl get pods\\
NAME ~ ~ ~ ~ ~ ~ ~ ~ ~ ~ ~ ~READY ~ ~ STATUS ~ ~RESTARTS ~ AGE\\
kafka-768ddf5564-xwgvh ~ ~ ~1/1 ~ ~ ~ Running ~ 0 ~ ~ ~ ~ ~31s\\
zookeeper-f798cc548-mgkpn ~ 1/1 ~ ~ ~ Running ~ 0 ~ ~ ~ ~
~1m\\[2\baselineskip]kubectl get services\\
NAME ~ ~ ~ ~TYPE ~ ~ ~ ~CLUSTER-IP ~ ~ ~EXTERNAL-IP ~ PORT(S) ~ ~ AGE\\
kafka ~ ~ ~ ClusterIP ~ 10.105.19.124 ~ \textless{}none\textgreater{} ~
~ ~ ~9092/TCP ~ ~6s\\
zookeeper ~ ClusterIP ~ 10.97.110.124 ~ \textless{}none\textgreater{} ~
~ ~ ~32181/TCP ~ 2m

      \vspace{\dp0}
      } \end{minipage} \\
      \\ \cdashline{2-3}

      & \begin{minipage}[t]{2cm}{Actual\\ Result}\end{minipage}   & 
      \begin{minipage}[t]{13cm}{\footnotesize
      
      \vspace{\dp0}
      } \end{minipage} \\
      \\ \cdashline{2-3}

      & Status          & Pass \\ \hline

      6 & Description &

      \begin{minipage}[t]{13cm}{\footnotesize
      Start a consumer that monitors the full stream and logs a deserialized
version of every Nth packet:\\

\begin{verbatim}
kubectl create -f consumerall-deployment.yaml
\end{verbatim}

      \vspace{\dp0}
      } \end{minipage} \\
      \\ \cdashline{2-3}

      & Expected Result & 

      \begin{minipage}[t]{13cm}{\footnotesize
      Runs without error

      \vspace{\dp0}
      } \end{minipage} \\
      \\ \cdashline{2-3}

      & \begin{minipage}[t]{2cm}{Actual\\ Result}\end{minipage}   & 
      \begin{minipage}[t]{13cm}{\footnotesize
      
      \vspace{\dp0}
      } \end{minipage} \\
      \\ \cdashline{2-3}

      & Status          & Pass \\ \hline

      7 & Description &

      \begin{minipage}[t]{13cm}{\footnotesize
      \begin{verbatim}
Start a producer that reads alert packets from disk and loads them into the Kafka queue:
\end{verbatim}

\begin{verbatim}
kubectl create -f sender-deployment.yaml
\end{verbatim}

      \vspace{\dp0}
      } \end{minipage} \\
      \\ \cdashline{2-3}

      & Expected Result & 

      \begin{minipage}[t]{13cm}{\footnotesize
      Runs without error

      \vspace{\dp0}
      } \end{minipage} \\
      \\ \cdashline{2-3}

      & \begin{minipage}[t]{2cm}{Actual\\ Result}\end{minipage}   & 
      \begin{minipage}[t]{13cm}{\footnotesize
      
      \vspace{\dp0}
      } \end{minipage} \\
      \\ \cdashline{2-3}

      & Status          & Pass \\ \hline

      8 & Description &

      \begin{minipage}[t]{13cm}{\footnotesize
      Determine the name of the alert sender pod with\\[2\baselineskip]kubectl
get pods\\[2\baselineskip]Examine output log
files.\\[2\baselineskip]kubectl logs \textless{}pod
name\textgreater{}\\[2\baselineskip]Verify that alerts are being sent
within 40 seconds by subtracting the timing measurements.

      \vspace{\dp0}
      } \end{minipage} \\
      \\ \cdashline{2-3}

      & Expected Result & 

      \begin{minipage}[t]{13cm}{\footnotesize
      Similar to\\[2\baselineskip]kubectl logs sender-7d6f98586f-nhwfj\\
visit: 1570. ~ ~ time: 1530588618.0313473\\
visits finished: 1 ~ ~ ~time: 1530588653.5614944\\
visit: 1571. ~ ~ time: 1530588657.0087624\\
visits finished: 2 ~ ~ ~time: 1530588692.506188\\
visit: 1572. ~ ~ time: 1530588696.0051727\\
visits finished: 3 ~ ~ ~time: 1530588731.5900314\\[3\baselineskip]

      \vspace{\dp0}
      } \end{minipage} \\
      \\ \cdashline{2-3}

      & \begin{minipage}[t]{2cm}{Actual\\ Result}\end{minipage}   & 
      \begin{minipage}[t]{13cm}{\footnotesize
      visit: 1570. ~ ~ time: 1530674964.0035944\\
visits finished: 1 ~ ~ ~time: 1530675000.5279133\\
visit: 1571. ~ ~ time: 1530675003.00461\\
visits finished: 2 ~ ~ ~time: 1530675039.5390277\\
visit: 1572. ~ ~ time: 1530675042.005049\\
visits finished: 3 ~ ~ ~time: 1530675078.4921534\\
visit: 1573. ~ ~ time: 1530675081.0037167\\[3\baselineskip]

      \vspace{\dp0}
      } \end{minipage} \\
      \\ \cdashline{2-3}

      & Status          & Pass \\ \hline

      9 & Description &

      \begin{minipage}[t]{13cm}{\footnotesize
      Determine the name of the consumer pod with\\[2\baselineskip]kubectl get
pods\\[2\baselineskip]Examine output log files.\\[2\baselineskip]kubectl
logs \textless{}pod name\textgreater{}\\[2\baselineskip]The packet log
should show deserialized alert packets with contents matching the input
packets.\\[2\baselineskip]

      \vspace{\dp0}
      } \end{minipage} \\
      \\ \cdashline{2-3}

      & Expected Result & 

      \begin{minipage}[t]{13cm}{\footnotesize
      Similar to \{'alertId': 12132024420, `l1dbId': 71776805594116,
`diaSource': \{'diaSourceId':\\
73499448928374785, `ccdVisitId': 2020011570, `diaObjectId':
71776805594116, 'ssO\\
bjectId': None, `parentDiaSourceId': None, `midPointTai': 59595.37041,
'filterNa\\
me': `y', `ra': 172.24912810036074, `decl': -80.64214929176521,
`ra\_decl\_Cov': \{\\
`raSigma': 0.0003428002819418907, `declSigma': 0.00027273103478364646,
'ra\_decl\_\\
Cov': 0.000628734880592674\}, `x': 2979.08837890625, `y':
3843.328857421875, 'x\_y\\
\_Cov': \{'xSigma': 0.6135467886924744, `ySigma': 0.77132648229599,
`x\_y\_Cov': 0.0\\
007463791407644749\}, `apFlux': None, `apFluxErr': None, `snr':
0.366516500711441\\
04, `psFlux': 7.698232025177276e-07, `psRa': None, `psDecl': None,
`ps\_Cov': Non\\
e, `psLnL': None, `psChi2': None, `psNdata': None, `trailFlux': None,
`trailRa':\\
etc.

      \vspace{\dp0}
      } \end{minipage} \\
      \\ \cdashline{2-3}

      & \begin{minipage}[t]{2cm}{Actual\\ Result}\end{minipage}   & 
      \begin{minipage}[t]{13cm}{\footnotesize
      ​​​​topic:full-stream, partition:0, status:end, offset:0, key:None,
time:1530674576.589\\
\{'alertId': 12132024420, `l1dbId': 71776805594116, `diaSource':
\{'diaSourceId': 73499448928374785, `ccdVisitId': 2020011570,
`diaObjectId': 71776805594116, `ssObjectId': None, `parentDiaSourceId':
None, `midPointTai': 59595.37041, `filterName': `y', `ra':
172.24912810036074, `decl': -80.64214929176521, `ra\_decl\_Cov':
\{'raSigma': 0.0003428002819418907, `declSigma': 0.00027273103478364646,
`ra\_decl\_Cov': 0.000628734880592674\}, `x': 2979.08837890625, `y':
3843.328857421875, `x\_y\_Cov': \{'xSigma': 0.6135467886924744,
`ySigma': 0.77132648229599, `x\_y\_Cov': 0.0007463791407644749\},
`apFlux': None, `apFluxErr': None, `snr': 0.36651650071144104, `psFlux':
7.698232025177276e-07, `psRa': None, `psDecl': None, `ps\_Cov': None,
`psLnL': None, `psChi2': None, `psNdata': None, `trailFlux': None,
`trailRa': None, `trailDecl': None, `trailLength': None, `trailAngle':
None, `trail\_Cov': None, `trailLnL': None, `trailChi2': None,
`trailNdata': None, `dipMeanFlux': None, `dipFluxDiff': None, `dipRa':
None, `dipDecl': None, `dipLength': None, `dipAngle': None, `dip\_Cov':
None, `dipLnL': None, `dipChi2': None, `dipNdata': None, `totFlux':
7.267262844834477e-06, `totFluxErr': 1.990927557926625e-06, `diffFlux':
7.698232025177276e-07, `diffFluxErr': 2.1003779693273827e-06, `fpBkgd':
None, `fpBkgdErr': None, `ixx': None, `iyy': None, `ixy': None,
`i\_cov': None, `ixxPSF': None, `iyyPSF': None, `ixyPSF': None,
`extendedness': None, `spuriousness': None, `flags': 516\},
`prv\_diaSources': None, `diaObject': \{'diaObjectId': 71776805594116,
`ra': 172.249128110081, `decl': -80.64214929171955, `ra\_decl\_Cov':
\{'raSigma': 0.0003098504093941301, `declSigma': 0.0001309745421167463,
`ra\_decl\_Cov': 0.0007125047268345952\}, `radecTai': 59595.139501,
`pmRa': -9.0, `pmDecl': -0.25999999046325684, `parallax':
1.1651986837387085, `pm\_parallax\_Cov': \{'pmRaSigma': 0.0,
`pmDeclSigma': 0.0, `parallaxSigma': 0.0, `pmRa\_pmDecl\_Cov': 0.0,
`pmRa\_parallax\_Cov': 0.0, `pmDecl\_parallax\_Cov': 0.0\},
`pmParallaxLnL': 0.290818989276886, `pmParallaxChi2': 0.399796724319458,
`pmParallaxNdata': 0, `uPSFluxMean': None, `uPSFluxMeanErr': None,
`uPSFluxSigma': None, `uPSFluxChi2': None, `uPSFluxNdata': None,
`gPSFluxMean': None, `gPSFluxMeanErr': None, `gPSFluxSigma': None,
`gPSFluxChi2': None, `gPSFluxNdata': None, `rPSFluxMean': None,
`rPSFluxMeanErr': None, `rPSFluxSigma': None, `rPSFluxChi2': None,
`rPSFluxNdata': None, `iPSFluxMean': None, `iPSFluxMeanErr': None,
`iPSFluxSigma': None, `iPSFluxChi2': None, `iPSFluxNdata': None,
`zPSFluxMean': None, `zPSFluxMeanErr': None, `zPSFluxSigma': None,
`zPSFluxChi2': None, `zPSFluxNdata': None, `yPSFluxMean': None,
`yPSFluxMeanErr': None, `yPSFluxSigma': None, `yPSFluxChi2': None,
`yPSFluxNdata': None, `uFPFluxMean': None, `uFPFluxMeanErr': None,
`uFPFluxSigma': None, `gFPFluxMean': None, `gFPFluxMeanErr': None,
`gFPFluxSigma': None, `rFPFluxMean': None, `rFPFluxMeanErr': None,
`rFPFluxSigma': None, `iFPFluxMean': None, `iFPFluxMeanErr': None,
`iFPFluxSigma': None, `zFPFluxMean': None, `zFPFluxMeanErr': None,
`zFPFluxSigma': None, `yFPFluxMean': None, `yFPFluxMeanErr': None,
`yFPFluxSigma': None, `uLcPeriodic': None, `gLcPeriodic': None,
`rLcPeriodic': None, `iLcPeriodic': None, `zLcPeriodic': None,
`yLcPeriodic': None, `uLcNonPeriodic': None, `gLcNonPeriodic': None,
`rLcNonPeriodic': None, `iLcNonPeriodic': None, `zLcNonPeriodic': None,
`yLcNonPeriodic': None, `nearbyObj1': None, `nearbyObj1Dist': None,
`nearbyObj1LnP': None, `nearbyObj2': None, `nearbyObj2Dist': None,
`nearbyObj2LnP': None, `nearbyObj3': None, `nearbyObj3Dist': None,
`nearbyObj3LnP': None, `flags': 873\}, `ssObject': None, `diaObjectL2':
None, `diaSourcesL2': None\}\\[3\baselineskip]

      \vspace{\dp0}
      } \end{minipage} \\
      \\ \cdashline{2-3}

      & Status          & Pass \\ \hline

    \end{longtable}


    \paragraph{Test Case LVV-T218 }\mbox{}\\

Open  \href{https://jira.lsstcorp.org/secure/Tests.jspa#/testCase/LVV-T218}{\textit{ LVV-T218 } }
test case in Jira.

    This test will demonstrate the ``mini-broker'' filtering service that
returns a subset of alerts from the full stream identified by
user-provided filters.\\[2\baselineskip]Specifically, this will
demonstrate that:\\

\begin{itemize}
\tightlist
\item
  The filtering service can retrieve alerts from the full alert stream
  and filter them according to their contents; ~ ~
\item
  The filtered subset can be delivered to science users.
\end{itemize}


    {\bf Preconditions}:\\
    Input data: A sample of Avro-formatted alert packets derived from LSST
simulations corresponding to one night of simulated LSST observing.


    Execution status: {\bf Conditional Pass }

    Final comment:\\Missing git-lfs in step 2 and 3. Used docker image instead. Last step
result deviated sligthly from the expected result for one consumer.



    Detailed step results:

    \begin{longtable}{p{1cm}p{2cm}p{13cm}}
    \hline
    {Step} & \multicolumn{2}{c}{Description, Results and Status}\\ \hline
      1 & Description &

      \begin{minipage}[t]{13cm}{\footnotesize
      Download Kafka Docker image from
https://github.com/lsst-dm/alert\_stream.

      \vspace{\dp0}
      } \end{minipage} \\
      \\ \cdashline{2-3}

      & Expected Result & 

      \begin{minipage}[t]{13cm}{\footnotesize
      Runs without error

      \vspace{\dp0}
      } \end{minipage} \\
      \\ \cdashline{2-3}

      & \begin{minipage}[t]{2cm}{Actual\\ Result}\end{minipage}   & 
      \begin{minipage}[t]{13cm}{\footnotesize
      
      \vspace{\dp0}
      } \end{minipage} \\
      \\ \cdashline{2-3}

      & Status          & Pass \\ \hline

      2 & Description &

      \begin{minipage}[t]{13cm}{\footnotesize
      Change to the alert\_stream directory and build the docker image.\\

\begin{verbatim}
docker build -t "lsst-kub001:5000/alert_stream"
\end{verbatim}

      \vspace{\dp0}
      } \end{minipage} \\
      \\ \cdashline{2-3}

      & Expected Result & 

      \begin{minipage}[t]{13cm}{\footnotesize
      Runs without error

      \vspace{\dp0}
      } \end{minipage} \\
      \\ \cdashline{2-3}

      & \begin{minipage}[t]{2cm}{Actual\\ Result}\end{minipage}   & 
      \begin{minipage}[t]{13cm}{\footnotesize
      Did not execute because of missing git-lfs; used pre-built Docker image
instead

      \vspace{\dp0}
      } \end{minipage} \\
      \\ \cdashline{2-3}

      & Status          & Conditional Pass \\ \hline

      3 & Description &

      \begin{minipage}[t]{13cm}{\footnotesize
      Register it with Kubernetes\\[2\baselineskip]docker push
lsst-kub001:5000/alert\_stream

      \vspace{\dp0}
      } \end{minipage} \\
      \\ \cdashline{2-3}

      & Expected Result & 

      \begin{minipage}[t]{13cm}{\footnotesize
      Runs without error

      \vspace{\dp0}
      } \end{minipage} \\
      \\ \cdashline{2-3}

      & \begin{minipage}[t]{2cm}{Actual\\ Result}\end{minipage}   & 
      \begin{minipage}[t]{13cm}{\footnotesize
      Did not execute because of missing git-lfs; used pre-built Docker image
instead

      \vspace{\dp0}
      } \end{minipage} \\
      \\ \cdashline{2-3}

      & Status          & Conditional Pass \\ \hline

      4 & Description &

      \begin{minipage}[t]{13cm}{\footnotesize
      From the alert\_stream/kubernetes directory, start Kafka and
Zookeeper:\\[2\baselineskip]

\begin{verbatim}
kubectl create -f zookeeper-service.yaml
kubectl create -f zookeeper-deployment.yaml
kubectl create -f kafka-deployment.yaml
kubectl create -f kafka-service.yaml
\end{verbatim}

(use kubectl get pods/services between each command to check status;
wait until each is ``Running'' before starting the next
command)\\[2\baselineskip]

      \vspace{\dp0}
      } \end{minipage} \\
      \\ \cdashline{2-3}

      & Expected Result & 

      \begin{minipage}[t]{13cm}{\footnotesize
      Runs without error

      \vspace{\dp0}
      } \end{minipage} \\
      \\ \cdashline{2-3}

      & \begin{minipage}[t]{2cm}{Actual\\ Result}\end{minipage}   & 
      \begin{minipage}[t]{13cm}{\footnotesize
      
      \vspace{\dp0}
      } \end{minipage} \\
      \\ \cdashline{2-3}

      & Status          & Pass \\ \hline

      5 & Description &

      \begin{minipage}[t]{13cm}{\footnotesize
      Confirm Kafka and Zookeeper are listed when
running\\[2\baselineskip]kubectl get
pods\\[2\baselineskip]and\\[2\baselineskip]kubectl get services

      \vspace{\dp0}
      } \end{minipage} \\
      \\ \cdashline{2-3}

      & Expected Result & 

      \begin{minipage}[t]{13cm}{\footnotesize
      Output should be similar to:\\[2\baselineskip]kubectl get pods\\
NAME ~ ~ ~ ~ ~ ~ ~ ~ ~ ~ ~ ~READY ~ ~ STATUS ~ ~RESTARTS ~ AGE\\
kafka-768ddf5564-xwgvh ~ ~ ~1/1 ~ ~ ~ Running ~ 0 ~ ~ ~ ~ ~31s\\
zookeeper-f798cc548-mgkpn ~ 1/1 ~ ~ ~ Running ~ 0 ~ ~ ~ ~
~1m\\[2\baselineskip]kubectl get services\\
NAME ~ ~ ~ ~TYPE ~ ~ ~ ~CLUSTER-IP ~ ~ ~EXTERNAL-IP ~ PORT(S) ~ ~ AGE\\
kafka ~ ~ ~ ClusterIP ~ 10.105.19.124 ~ \textless{}none\textgreater{} ~
~ ~ ~9092/TCP ~ ~6s\\
zookeeper ~ ClusterIP ~ 10.97.110.124 ~ \textless{}none\textgreater{} ~
~ ~ ~32181/TCP ~ 2m

      \vspace{\dp0}
      } \end{minipage} \\
      \\ \cdashline{2-3}

      & \begin{minipage}[t]{2cm}{Actual\\ Result}\end{minipage}   & 
      \begin{minipage}[t]{13cm}{\footnotesize
      
      \vspace{\dp0}
      } \end{minipage} \\
      \\ \cdashline{2-3}

      & Status          & Pass \\ \hline

      6 & Description &

      \begin{minipage}[t]{13cm}{\footnotesize
      Start 100 consumers that consume the filtered streams and logs a
deserialized version of every Nth packet:\\[2\baselineskip]

\begin{verbatim}
kubectl create -f consumer1-deployment.yaml
kubectl create -f consumer2-deployment.yaml
kubectl create -f consumer3-deployment.yaml
kubectl create -f consumer4-deployment.yaml
kubectl create -f consumer5-deployment.yaml
kubectl create -f consumer6-deployment.yaml
kubectl create -f consumer7-deployment.yaml
kubectl create -f consumer8-deployment.yaml
kubectl create -f consumer9-deployment.yaml
kubectl create -f consumer10-deployment.yaml
\end{verbatim}

      \vspace{\dp0}
      } \end{minipage} \\
      \\ \cdashline{2-3}

      & Expected Result & 

      \begin{minipage}[t]{13cm}{\footnotesize
      Runs without error

      \vspace{\dp0}
      } \end{minipage} \\
      \\ \cdashline{2-3}

      & \begin{minipage}[t]{2cm}{Actual\\ Result}\end{minipage}   & 
      \begin{minipage}[t]{13cm}{\footnotesize
      
      \vspace{\dp0}
      } \end{minipage} \\
      \\ \cdashline{2-3}

      & Status          & Pass \\ \hline

      7 & Description &

      \begin{minipage}[t]{13cm}{\footnotesize
      Start 5 filter groups:\\

\begin{verbatim}
kubectl create -f filterer1-deployment.yaml
kubectl create -f filterer2-deployment.yaml
kubectl create -f filterer3-deployment.yaml
kubectl create -f filterer4-deployment.yaml
kubectl create -f filterer5-deployment.yaml
\end{verbatim}

      \vspace{\dp0}
      } \end{minipage} \\
      \\ \cdashline{2-3}

      & Expected Result & 

      \begin{minipage}[t]{13cm}{\footnotesize
      Runs without error

      \vspace{\dp0}
      } \end{minipage} \\
      \\ \cdashline{2-3}

      & \begin{minipage}[t]{2cm}{Actual\\ Result}\end{minipage}   & 
      \begin{minipage}[t]{13cm}{\footnotesize
      
      \vspace{\dp0}
      } \end{minipage} \\
      \\ \cdashline{2-3}

      & Status          & Pass \\ \hline

      8 & Description &

      \begin{minipage}[t]{13cm}{\footnotesize
      Start a producer that reads alert packets from disk and loads them into
the Kafka queue:\\[2\baselineskip]

\begin{verbatim}
kubectl create -f sender-deployment.yaml
\end{verbatim}

      \vspace{\dp0}
      } \end{minipage} \\
      \\ \cdashline{2-3}

      & Expected Result & 

      \begin{minipage}[t]{13cm}{\footnotesize
      Runs without error

      \vspace{\dp0}
      } \end{minipage} \\
      \\ \cdashline{2-3}

      & \begin{minipage}[t]{2cm}{Actual\\ Result}\end{minipage}   & 
      \begin{minipage}[t]{13cm}{\footnotesize
      
      \vspace{\dp0}
      } \end{minipage} \\
      \\ \cdashline{2-3}

      & Status          & Pass \\ \hline

      9 & Description &

      \begin{minipage}[t]{13cm}{\footnotesize
      Determine the name of the alert sender pod with\\[2\baselineskip]kubectl
get pods\\[2\baselineskip]Examine output log
files.\\[2\baselineskip]kubectl logs \textless{}pod
name\textgreater{}\\[2\baselineskip]Verify that alerts are being sent
within 40 seconds by subtracting the timing measurements.

      \vspace{\dp0}
      } \end{minipage} \\
      \\ \cdashline{2-3}

      & Expected Result & 

      \begin{minipage}[t]{13cm}{\footnotesize
      Similar to\\[2\baselineskip]kubectl logs sender-7d6f98586f-nhwfj\\
visit: 1570. ~ ~ time: 1530588618.0313473\\
visits finished: 1 ~ ~ ~time: 1530588653.5614944\\
visit: 1571. ~ ~ time: 1530588657.0087624\\
visits finished: 2 ~ ~ ~time: 1530588692.506188\\
visit: 1572. ~ ~ time: 1530588696.0051727\\
visits finished: 3 ~ ~ ~time: 1530588731.5900314\\[2\baselineskip]

      \vspace{\dp0}
      } \end{minipage} \\
      \\ \cdashline{2-3}

      & \begin{minipage}[t]{2cm}{Actual\\ Result}\end{minipage}   & 
      \begin{minipage}[t]{13cm}{\footnotesize
      visit: 1570. ~ ~ time: 1530675627.0249922\\
visits finished: 1 ~ ~ ~time: 1530675664.2054636\\
visit: 1571. ~ ~ time: 1530675666.00571\\
visits finished: 2 ~ ~ ~time: 1530675702.7181122\\
visit: 1572. ~ ~ time: 1530675705.0034182\\
visits finished: 3 ~ ~ ~time: 1530675741.6658716\\
visit: 1573. ~ ~ time: 1530675744.0031254\\
visits finished: 4 ~ ~ ~time: 1530675780.6261215\\
visit: 1574. ~ ~ time: 1530675783.0040474\\[3\baselineskip]

      \vspace{\dp0}
      } \end{minipage} \\
      \\ \cdashline{2-3}

      & Status          & Pass \\ \hline

      10 & Description &

      \begin{minipage}[t]{13cm}{\footnotesize
      Determine the name of the consumer pods with\\[2\baselineskip]kubectl
get pods\\[2\baselineskip]Examine output log
files.\\[2\baselineskip]kubectl logs \textless{}pod
name\textgreater{}\\[2\baselineskip]The packet log should show
deserialized alert packets with contents matching the input packets.

      \vspace{\dp0}
      } \end{minipage} \\
      \\ \cdashline{2-3}

      & Expected Result & 

      \begin{minipage}[t]{13cm}{\footnotesize
      Similar to\\[2\baselineskip]\{'alertId': 12132024420, `l1dbId':
71776805594116, `diaSource': \{'diaSourceId':\\
73499448928374785, `ccdVisitId': 2020011570, `diaObjectId':
71776805594116, 'ssO\\
bjectId': None, `parentDiaSourceId': None, `midPointTai': 59595.37041,
'filterNa\\
me': `y', `ra': 172.24912810036074, `decl': -80.64214929176521,
`ra\_decl\_Cov': \{\\
`raSigma': 0.0003428002819418907, `declSigma': 0.00027273103478364646,
'ra\_decl\_\\
Cov': 0.000628734880592674\}, `x': 2979.08837890625, `y':
3843.328857421875, 'x\_y\\
\_Cov': \{'xSigma': 0.6135467886924744, `ySigma': 0.77132648229599,
`x\_y\_Cov': 0.0\\
007463791407644749\}, `apFlux': None, `apFluxErr': None, `snr':
0.366516500711441\\
04, `psFlux': 7.698232025177276e-07, `psRa': None, `psDecl': None,
`ps\_Cov': Non\\
e, `psLnL': None, `psChi2': None, `psNdata': None, `trailFlux': None,
`trailRa':\\
etc.

      \vspace{\dp0}
      } \end{minipage} \\
      \\ \cdashline{2-3}

      & \begin{minipage}[t]{2cm}{Actual\\ Result}\end{minipage}   & 
      \begin{minipage}[t]{13cm}{\footnotesize
      kubectl logs consumer1-6cdf9b57f4-swrpk \textbar{} head -n 2\\
topic:Filter001, partition:0, status:end, offset:0, key:None,
time:1530675384.799\\
\{'alertId': 12132024334, `l1dbId': 71776752985092, `diaSource':
\{'diaSourceId': 73499395056734209, `ccdVisitId': 1020011570,
`diaObjectId': 71776752985092,
`ssObjectId':\ldots{}\\[2\baselineskip]consumer 2:\\
topic:Filter011, partition:0, status:end, offset:0, key:None,
time:1530675474.542\\
\{'alertId': 12132024334, `l1dbId': 71776752985092, `diaSource':
\{'diaSourceId': 73499395056734209, `ccdVisitId': 1020011570,
`diaObjectId': 71776752985092, `ssObjectId': None,
\ldots{}\\[2\baselineskip](similar results for consumers 3-5,
7-10)\\[2\baselineskip]consumer 6:\\
\%3\textbar{}1530675391.040\textbar{}FAIL\textbar{}rdkafka\#consumer-1\textbar{}
{[}thrd:kafka.alerts-lsst.svc.cluster.local:9092/bootstrap{]}:
kafka.alerts-lsst.svc.cluster.local:9092/bootstrap: Failed to resolve
`kafka.alerts-lsst.svc.cluster.local:9092': Temporary failure in name
resolution\\
\%3\textbar{}1530675391.040\textbar{}ERROR\textbar{}rdkafka\#consumer-1\textbar{}
{[}thrd:kafka.alerts-lsst.svc.cluster.local:9092/bootstrap{]}:
kafka.alerts-lsst.svc.cluster.local:9092/bootstrap: Failed to resolve
`kafka.alerts-lsst.svc.cluster.local:9092': Temporary failure in name
resolution\\
\%3\textbar{}1530675391.040\textbar{}ERROR\textbar{}rdkafka\#consumer-1\textbar{}
{[}thrd:kafka.alerts-lsst.svc.cluster.local:9092/bootstrap{]}: 1/1
brokers are down

      \vspace{\dp0}
      } \end{minipage} \\
      \\ \cdashline{2-3}

      & Status          & Conditional Pass \\ \hline

    \end{longtable}


\input{appendix.tex}
\end{document}
